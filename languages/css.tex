\subsubsection{Inline CSS}
Here's how you would set your h2 element's text color to blue:

\begin{codeBlock}{html}{Inline CSS}
<h2 style="color: blue;">CatPhotoApp</h2>
\end{codeBlock}

\begin{tcolorbox}
It is a good practice to end inline style declarations with a ; .
\end{tcolorbox}

\subsubsection{Global CSS}
\begin{codeBlock}{html}{Changing global styles}
<style>
  h2 {
    color: red;
  }
</style>
\end{codeBlock}
Note that it's important to have both opening and closing curly braces (\{ and \}) around each element's style rule(s). You also need to make sure that your element's style definition is between the opening and closing style tags. Finally, be sure to add a semicolon to the end of each of your element's style rules.

\subsubsection{CSS Classes}
\begin{codeBlock}{html}{CSS Clss Declarations}
<style>
  .blue-text {
    color: blue;
  }
</style>
\end{codeBlock}

You can see that we've created a CSS class called blue-text within the \mintinline{html}{<style>} tag. You can apply a class to an HTML element like this: \mintinline{html}{<h2 class="blue-text">CatPhotoApp</h2>}. Note that in your CSS style element, class names start with a period. In your HTML elements' class attribute, the class name does not include the period.

\subsubsection{Font Degradation}
For example, if you wanted an element to use the Helvetica font, but degrade to the sans-serif font when Helvetica isn't available, you will specify it as follows:
\begin{codeBlock}{css}{Font Degradation}
p {
  font-family: Helvetica, sans-serif;
}
\end{codeBlock}
Generic font family names are not case-sensitive. Also, they do not need quotes because they are CSS keywords.

\subsubsection{id Attribute}
In addition to classes, each HTML element can also have an id attribute.

There are several benefits to using id attributes: You can use an id to style a single element and later you'll learn that you can use them to select and modify specific elements with JavaScript.

id attributes should be unique. Browsers won't enforce this, but it is a widely agreed upon best practice. So please don't give more than one element the same id attribute.

Here's an example of how you give a h2 element the id of cat-photo-app:

\begin{codeBlock}{html}{Setting an id Attribute}

<h2 id="cat-photo-app">
\end{codeBlock}

\subsubsection{Styles Using id}
One cool thing about id attributes is that, like classes, you can style them using CSS.

However, an id is not reusable and should only be applied to one element. An id also has a higher specificity (importance) than a class so if both are applied to the same element and have conflicting styles, the styles of the id will be applied.

Here's an example of how you can take your element with the id attribute of cat-photo-element and give it the background color of green. In your style element:
\begin{codeBlock}{css}{id Attribute Style}
#cat-photo-element {
  background-color: green;
}
\end{codeBlock}

\subsubsection{Alignment, margin, and Padding}
Three important properties control the space that surrounds each HTML element: padding, border, and margin.

An element's \emph{padding} controls the amount of space between the element's content and its border.

An element's \emph{margin} controls the amount of space between an element's border and surrounding elements.

If you set an element's \emph{margin} to a negative value, the element will grow larger.

\subsubsection{Clockwise Notation for Padding}
Instead of specifying an element's padding-top, padding-right, padding-bottom, and padding-left properties individually, you can specify them all in one line, like this:
\mint{css}{padding: 10px 20px 10px 20px;
}
These four values work like a clock: top, right, bottom, left, and will produce the exact same result as using the side-specific padding instructions.

\subsubsection{Measurement Units}
The two main types of length units are absolute and relative. Absolute units tie to physical units of length. For example, in and mm refer to inches and millimeters, respectively. Absolute length units approximate the actual measurement on a screen, but there are some differences depending on a screen's resolution.

Relative units, such as em or rem, are relative to another length value. For example, em is based on the size of an element's font. If you use it to set the font-size property itself, it's relative to the parent's font-size.

\href{https://www.w3.org/Style/Examples/007/units.en.html}{This article} written by Bert Bos, a member of the W3C board, goes more in-depth about absolute and relative units in CSS. According to Bert, the uses of each unit differs by medium.

\begin{table}[h]
    \centering
    \begin{tabular}{c|c|c|c}
         &Recommended & Occasional Use & Not Recommended\\
         \hline\hline
         Screen & em,px, \%&ex&pt, cm, mm, in, pc \\
         Print& em, cm, mm, in, pt, pc, \%&px, ex&\\
    \end{tabular}
    \caption{Recommended Usage of Units}
\end{table}
Absolute units (cm, mm, in, pt, and pc) should be avoided in uses other than print. The reason being that an inch on a desktop screen looks very small when compared to an inch on a mobile phone. In this case, using a relative unit (em, ex, px, \%) is better.

The magic unit of CSS, the px, is a often a good unit to use, especially if the style requires alignment of text to images, or simply because anything that is 1px wide or a multiple of 1px is guaranteed to look sharp.

But for font sizes it is even better to use em. The idea is (1) to not set the font size of the BODY element (in HTML), but use the default size of the device, because that is a size that the reader can comfortably read; and (2) express font sizes of other elements in em: H1 {font-size: 2.5em} to make the H1 2$\frac{1}{2}$ times as big as the normal, body font.

The only place where you could use pt (or cm or in) for setting a font size is in style sheets for print, if you need to be sure the printed font is exactly a certain size. But even there using the default font size is usually better.

\begin{tcolorbox}

The px unit thus shields you from having to know the resolution of the device. Whether the output is 96dpi, 100dpi, 220dpi or 1800dpi, a length expressed as a whole number of px always looks good and very similar across all devices. But what if you do want to know the resolution of the device, e.g., to know if it is safe to use a 0.5px line? The answer is to check the resolution via Media Queries. Explaining Media Queries is out of scope for this article, but here is a small example:
\end{tcolorbox}

\begin{codeBlock}{css}{Media Query Example}
div.mybox { border: 2px solid }
@media (min-resolution: 2dppx) {
  /* Media with 2 or more dots per px */
  div.mybox { border: 1.5px solid }
}
\end{codeBlock}

\subsubsection{Overriding CSS}
A class overrides the body's style properties. Class B declared after class A overrides the overlapping effects of class A regardless of order of use in html. For example:

\begin{codeBlock}{html}{Overriding CSS}
<style>
  body {
    background-color: black;
    font-family: monospace;
    color: green;
  }
  .pink-text {
    color: pink;
  }
  .blue-text {
    color: blue;
  }
  #orange-text {
    color: orange;
  }
</style>

<h1 class="pink-text blue-text" id="orange-text">Hello World!</h1>
\end{codeBlock}

Because \mintinline{css}{.blue-text} was declared in \mintinline{html}{<style></style>} after \mintinline{css}{.pink-text}, it takes priority. However, the id will take precedence over the class and override it's effects. Furthermore, inline styles override all stye declarations. Lastly, the most powerful overrides is that of \emph{!important}. Adding this to any CSS declaration ensures that that style is implemented.

\subsubsection{Variables}
Create a custom CSS Variable

To create a CSS variable, you just need to give it a name with two hyphens in front of it and assign it a value like this:
\begin{codeBlock}{css}{Setting a Variable}
--penguin-skin: gray;
\end{codeBlock}

This will create a variable named --penguin-skin and assign it the value of gray. Now you can use that variable elsewhere in your CSS to change the value of other elements to gray.

\subsubsection{Fallback Values for Variables}

Despite the following code having a typo, the browser displays a black color for the background because of the fallback.
\begin{codeBlock}{css}{Fallbacks}
  .penguin {
    --penguin-skin: black;
    --penguin-belly: gray;
    --penguin-beak: yellow;
  }

  .penguin-top {
    top: 10%;
    left: 25%;

    background: var(--pengiun-skin, black);
  }
\end{codeBlock}

\subsection{Applied Visual Design}
\subsubsection{Text-Align}
Text is often a large part of web content. CSS has several options for how to align it with the text-align property.

\mintinline{css}{text-align:} justify; causes all lines of text except the last line to meet the left and right edges of the line box.

\mintinline{css}{text-align:} center; centers the text

\mintinline{css}{text-align:} right; right-aligns the text

And \mintinline{css}{text-align: left;} (the default) left-aligns the text.

\subsubsection{Bold, Underline, Italicize}
To make text bold, you can use the strong tag. This is often used to draw attention to text and symbolize that it is important. With the strong tag, the browser applies the CSS of font-weight: bold; to the element.
\mintinline{html}{<strong></strong>}

To underline text, you can use the u tag. This is often used to signify that a section of text is important, or something to remember. With the u tag, the browser applies the CSS of text-decoration: underline; to the element. The u tag is equivalent to the browser applying the CSS of \mintinline{css}{text-decoration:underline;}

\mintinline{html}{<u></u>}

To emphasize text, you can use the em tag. This displays text as italicized, as the browser applies the CSS of \mintinline{css}{font-style: italic;} to the element.

\mintinline{html}{<em></em>}

\subsubsection{Strikethrough}
To strikethrough text, which is when a horizontal line cuts across the characters, you can use the s tag. It shows that a section of text is no longer valid. With the s tag, the browser applies the CSS of \mintinline[breaklines]{css}{text-decoration: line-through;} to the element.
\mint{html}{<s></s>}

\subsubsection{Box Shadow}


The box-shadow property applies one or more shadows to an element.

The box-shadow property takes values for
\begin{itemize}
    \item \mintinline{css}{offset-x} (how far to push the shadow horizontally from the element),
    \item \mintinline{css}{offset-y} (how far to push the shadow vertically from the element),
    \item \mintinline{css}{blur-radius},
    \item \mintinline{css}{spread-radius} and
    \item \mintinline{css}{color}, in that order.
\end{itemize}
The blur-radius and spread-radius values are optional.

Multiple box-shadows can be created by using commas to separate properties of each box-shadow element.

Here's an example of the CSS to create multiple shadows with some blur, at mostly-transparent black colors:

\mint[breaklines]{css}{box-shadow: 0 10px 20px rgba(0,0,0,0.19), 0 6px 6px rgba(0,0,0,0.23);}

\subsubsection{Hover State of an Anchor Tag}
A pseudo-class is a keyword that can be added to selectors, in order to select a specific state of the element.

For example, the styling of an anchor tag can be changed for its hover state using the :hover pseudo-class selector. Here's the CSS to change the color of the anchor tag to red during its hover state:

\mint{css}{
a:hover {
color: red;
}
}

\subsubsection{Absolute Position}


The next option for the CSS position property is absolute, which locks the element in place relative to its parent container. Unlike the relative position, this removes the element from the normal flow of the document, so surrounding items ignore it. The CSS offset properties (top or bottom and left or right) are used to adjust the position.

One nuance with absolute positioning is that it will be locked relative to its closest positioned ancestor. If you forget to add a position rule to the parent item, (this is typically done using position: relative;), the browser will keep looking up the chain and ultimately default to the body tag.

\subsubsection{Fixed Position}


The next layout scheme that CSS offers is the fixed position, which is a type of absolute positioning that locks an element relative to the browser window. Similar to absolute positioning, it's used with the CSS offset properties and also removes the element from the normal flow of the document. Other items no longer "realize" where it is positioned, which may require some layout adjustments elsewhere.

One key difference between the fixed and absolute positions is that an element with a fixed position won't move when the user scrolls.

\subsubsection{Float}
The next positioning tool does not actually use position, but sets the float property of an element. Floating elements are removed from the normal flow of a document and pushed to either the left or right of their containing parent element. It's commonly used with the width property to specify how much horizontal space the floated element requires.
\mint{css}{float:left;}
\mint{css}{float:right;}

\subsubsection{Overlap Heirarchy with Z-index}
When elements are positioned to overlap (i.e. using \mintinline[breaklines]{css}{position: absolute | relative | fixed | sticky}), the element coming later in the HTML markup will, by default, appear on the top of the other elements. However, the z-index property can specify the order of how elements are stacked on top of one another. It must be an integer (i.e. a whole number and not a decimal), and higher values for the z-index property of an element move it higher in the stack than those with lower values.

\mint{css}{z-index:0...x}

\subsubsection{Horizontal Centering Using Auto Margins}


Another positioning technique is to center a block element horizontally. One way to do this is to set its margin to a value of auto.

This method works for images, too. Images are inline elements by default, but can be changed to block elements when you set the display property to block.

\mint{css}{margin:auto;}

\subsubsection{Tertiary Colors}
Computer monitors and device screens create different colors by combining amounts of red, green, and blue light. This is known as the RGB additive color model in modern color theory. Red (R), green (G), and blue (B) are called primary colors. Mixing two primary colors creates the secondary colors cyan (G + B), magenta (R + B) and yellow (R + G). You saw these colors in the Complementary Colors challenge. These secondary colors happen to be the complement to the primary color not used in their creation, and are opposite to that primary color on the color wheel. For example, magenta is made with red and blue, and is the complement to green.

Tertiary colors are the result of combining a primary color with one of its secondary color neighbors. For example, within the RGB color model, red (primary) and yellow (secondary) make orange (tertiary). This adds six more colors to a simple color wheel for a total of twelve.

There are various methods of selecting different colors that result in a harmonious combination in design. One example that can use tertiary colors is called the split-complementary color scheme. This scheme starts with a base color, then pairs it with the two colors that are adjacent to its complement. The three colors provide strong visual contrast in a design, but are more subtle than using two complementary colors.

Here are three colors created using the split-complement scheme:
\begin{table}[h]
    \centering
    \begin{tabular}{c|c}
         Color&Hex Code  \\
         \hline\hline
         Orange& \#FF7F00\\
         Cyan & \#00FFFF\\
         Raspberry&\#FF007F\\
    \end{tabular}
\end{table}

\subsubsection{Color Gradient}
Applying a color on HTML elements is not limited to one flat hue. CSS provides the ability to use color transitions, otherwise known as gradients, on elements. This is accessed through the background property's linear-gradient() function. Here is the general syntax:

\mint[breaklines]{css}{background: linear-gradient(gradient_direction, color 1, color 2, color 3, ...);}

The first argument specifies the direction from which color transition starts - it can be stated as a degree, where 90deg makes a horizontal gradient (from left to right) and 45deg makes a diagonal gradient (from bottom left to top right). The following arguments specify the order of colors used in the gradient.

Example:

\mint[breaklines]{css}{background: linear-gradient(90deg, red, yellow, rgb(204, 204, 255))}

\subsubsection{CSS Stripes Using Repeating Linear Gradients}


The repeating-linear-gradient() function is very similar to linear-gradient() with the major difference that it repeats the specified gradient pattern. repeating-linear-gradient() accepts a variety of values, but for simplicity, you'll work with an angle value and color stop values in this challenge.

The angle value is the direction of the gradient. Color stops are like width values that mark where a transition takes place, and are given with a percentage or a number of pixels.

In the example demonstrated in the code editor, the gradient starts with the color yellow at 0 pixels which blends into the second color blue at 40 pixels away from the start. Since the next color stop is also at 40 pixels, the gradient immediately changes to the third color green, which itself blends into the fourth color value red as that is 80 pixels away from the beginning of the gradient.

For this example, it helps to think about the color stops as pairs where every two colors blend together.

\mint[breaklines]{css}{
0px [yellow -- blend -- blue] 40px [green -- blend -- red] 80px}

If every two color stop values are the same color, the blending isn't noticeable because it's between the same color, followed by a hard transition to the next color, so you end up with stripes.

In other words, you can control the end location of a gradient by duplicating the color to the start location of the next color.

\mint[breaklines]{css}{background-color:repeating-linear-gradient(45deg, color1 start, color1 end, color2 start, color2 end)}

\subsubsection{Animations}

To animate an element, you need to know about the animation properties and the @keyframes rule. The animation properties control how the animation should behave and the @keyframes rule controls what happens during that animation.

The 8 animation properties are as follows:

\begin{itemize}
    \item \mintinline{css}{animation-name}

        Specifies a name for the \mintinline{css}{@keyframes} animation.
    \item \mintinline{css}{animation-duration}

        Defines how long an animation should take to complete one cycle.
    \item \mintinline{css}{animation-timing-function}

        Specifies the speed curve of an animation. i.e. Linear, ease, ease-in, ease-out, ease-in-out, etc.
    \item \mintinline{css}{animation-delay}

        Specifies a delay value for the start of an animation.
    \item \mintinline{css}{animation-iteration-count}

        Specifies the amount of times an animation should be played.
    \item \mintinline{css}{animation-direction}

        Specifies how an animation should be played. i.e. normal, reverse, alternate, alternate-reverse, etc.
    \item \mintinline{css}{animation-fill-mode}

        Specifies which style the element inherits when the animation is not playing. This can be before it starts, after it ends, or both. Forwards - retain style of last keyframe once animation is over. Backwards - get the style of the first keyframe and retain until delay is over. Both - follow rules for both forwards and backwards.
    \item \mintinline{css}{animation-play-state}

        Use this in JavaScript to control whether the animation is paused or playing.
\end{itemize}

Unless specified, the animation-duration property defaults to 0.

\mintinline{css}{@keyframes} is how to specify exactly what happens within the animation over the duration. This is done by giving CSS properties for specific "frames" during the animation, with percentages ranging from 0\% to 100\%. If you compare this to a movie, the CSS properties for 0\% is how the element displays in the opening scene. The CSS properties for 100\% is how the element appears at the end, right before the credits roll. Then CSS applies the magic to transition the element over the given duration to act out the scene. Here's an example to illustrate the usage of \mintinline{css}{@keyframes} and the animation properties:
\begin{codeBlock}{css}{Example Animation}
#anim {
  animation-name: colorful;
  animation-duration: 3s;
}

@keyframes colorful {
  0% {
    background-color: blue;
  }
  100% {
    background-color: yellow;
  }
}
\end{codeBlock}

\subsubsection{Movement Using CSS Animation}
When elements have a specified position, such as fixed or relative, the CSS offset properties right, left, top, and bottom can be used in animation rules to create movement.

\subsubsection{Infinite Animation Loops}

You can manually set the amount of iterations an animation does by using the \mintinline[breaklines]{css}{animation-iteration-count:x;} property. By setting the value of X to \emph{infinite}, you can force the animation to loop endlessly.

\subsubsection{Animate Elements at Variable Rates}


There are a variety of ways to alter the animation rates of similarly animated elements. So far, this has been achieved by applying an animation-iteration-count property and setting @keyframes rules.

To illustrate, the animation on the right consists of two stars that each decrease in size and opacity at the 20\% mark in the \mintinline[breaklines]{css}{@keyframes} rule, which creates the twinkle animation. You can change the \mintinline[breaklines]{css}{@keyframes} rule for one of the elements so the stars twinkle at different rates.

Animation rates can also be changed by altering the \mintinline{css}{animation-duration} of multiple elements.

\subsubsection{Bezier Curves in CSS}


The last challenge introduced the animation-timing-function property and a few keywords that change the speed of an animation over its duration. CSS offers an option other than keywords that provides even finer control over how the animation plays out, through the use of Bezier curves.

In CSS animations, Bezier curves are used with the \mintinline{css}{cubic-bezier} function. The shape of the curve represents how the animation plays out. The curve lives on a 1 by 1 coordinate system. The X-axis of this coordinate system is the duration of the animation (think of it as a time scale), and the Y-axis is the change in the animation.

The cubic-bezier function consists of four main points that sit on this 1 by 1 grid: p0, p1, p2, and p3. p0 and p3 are set for you - they are the beginning and end points which are always located respectively at the origin (0, 0) and (1, 1). You set the x and y values for the other two points, and where you place them in the grid dictates the shape of the curve for the animation to follow. This is done in CSS by declaring the x and y values of the p1 and p2 "anchor" points in the form: (x1, y1, x2, y2). Pulling it all together, here's an example of a Bezier curve in CSS code:
\mint[breaklines]{css}{
animation-timing-function: cubic-bezier(0.25, 0.25, 0.75, 0.75);
}
In the example above, the x and y values are equivalent for each point (x1 = 0.25 = y1 and x2 = 0.75 = y2), which if you remember from geometry class, results in a line that extends from the origin to point (1, 1). This animation is a linear change of an element during the length of an animation, and is the same as using the linear keyword. In other words, it changes at a constant speed.
