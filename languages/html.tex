\subsubsection{Input Labels}
It is considered best practice to set a \emph{for} attribute on the \emph{label} element, with a value that matches the value of the \emph{id} attribute of the input element. This allows assistive technologies to create a linked relationship between the \emph{label} and the related input element. For example:
\begin{codeBlock}{html}{Labels and inputs}
    <input id="indoor" type="radio" name="indoor-outdoor">
    <label for="indoor">Indoor</label>
\end{codeBlock}

Alternatively, we can nest \emph{input} elements within \emph{label} tags.
\begin{codeBlock}{html}{Alternative Code}
    <label for="indoor"> 
        <input id="indoor" type="radio" name="indoor-outdoor">Indoor 
    </label>
\end{codeBlock}

All related radio buttons should have the same \emph{name} attribute to create a radio button group. By creating a radio group, selecting any single radio button will automatically deselect the other buttons within the same group ensuring only one answer is provided by the user.

\begin{codeBlock}{html}{Label Groups}
<label for="indoor">
    <input id="indoor" type="radio" name="indoor-outdoor"> Indoor
</label>
<label for="outdoor">
    <input id="outdoor" type="radio" name="indoor-outdoor"> Outdoor
</label>    
\end{codeBlock}
Here, you have two radio inputs. When the user submits the form with the indoor option selected, the form data will include the line: indoor-outdoor=indoor. This is from the name and value attributes of the "indoor" input.

If you omit the value attribute, the submitted form data uses the default value, which is on. 

\begin{codeBlock}{html}{Input Values}

<label for="indoor">
  <input id="indoor" value="indoor" type="radio" name="indoor-outdoor">Indoor
</label>
<label for="outdoor">
  <input id="outdoor" value="outdoor" type="radio" name="indoor-outdoor">Outdoor
</label>
\end{codeBlock}

Then any markup with the content of the page (what displays for a user) would go into the body tag.

Metadata elements, such as link, meta, title, and style, typically go inside the head element.

Here's an example of a page's layout:
\begin{codeBlock}{html}{Page Layout}

<!DOCTYPE html>
<html>
  <head>
    <meta />
  </head>
  <body>
    <div>
    </div>
  </body>
</html>

\end{codeBlock}