\section{Basic Front-end Deployment - FE, BE, FS}
You should be able to do a basic website or frontend app deployment.
\begin{itemize}
    \item Static Hosting - Netlify, GitHub Pages, Heroku

    Netlify uses Git so all you have to do is push to your repository and you have continuous deployment using Netflify.
    \item CPanel Hosting - InMotion, Hostgator, Bluehost
    The good thing about CPanel hosting is that there is a lot included by default like e-mail hosting, SQL databases, STP accounts, etc. A VPS(Virtual Private Server) is suggested for CPanel and not shared hosting. 
\end{itemize}

Methods of Deploying:
There are various ways of getting your files onto your server.
\begin{itemize}
    \item Git - Continous deployment by pushing to a repo
    \item FTP/ SFTP - File Transfer Protocol/Secure File Transfer Protocol (Slow)
    \item SSH - Secure Shell (Terminal)
\end{itemize}
There are ways of hosting full stack applications - AWS, Digital Ocean, etc.

Some other things you will run into during a basic deployment: 
\begin{itemize}
    \item Domain Names - Namecheap, Google Domains, Enom
    
    If you use something like Netlify, you can have your main domain name point to Netlify and then setup your e-mail hosting somewhere like Namecheap.
    \item Email Hosting - Namecheap, Zoho Mail, CPanel.
    \item Let's Encrypt, Cloudflare, Namecheap
\end{itemize}

Basically, you want to know how to host a website, connect a domain name, setup a SSL. At this point, you are basically a Foundational Frontend Developer.

\section{Basics Conclusion- FE, FS}
At this point, you should be able to do the following: 
\begin{itemize}
    \item Setup a productive development environment.
    \item Write HTML, CSS, and JS.
    \item Use Sass and CSS framework (optional).
    \item Create responsive layouts.
    \item Build websites with some dynamic functionality and work with the DOM.
    \item Connect to 3rd party APIs with Fetch and understand basic HTTP.
    \item Use Git with GitHub or some other Git repo (Bitbucket, etc).
    \item Deploy and manage a website or small web app.
\end{itemize}

If you can consider these skills conquered, you are essentially a \uline{Foundational Front-end Developer}.

\section{What Now?}
At this point, you can take a couple different routes. You can sharpen your JS skills, start doing algorithms or more advanced JS. If you want to stick to front-end, move to a front-end framework like React. Or you can move to back-end.
