\section{HTML and CSS - FE, FS, \emph{BE}}
\begin{itemize}
    \item HTML 5 Page Structure \& Semantic Tags
    
    Header section, try not to use divs, etc.
    \item Basic CSS Styling
    
    Colors, fonts, padding and margins, etc
    \item Positioning
    
    Absolute, relative, fixed, etc.
    \item Alignment (Flexbox \& CSS Grid)

    At least learn flexbox, if not both.
    \item Transitions and Animation
    
    Very important for Front-end devs. A lot of UI's have some kind of animation involved.
    \item Responsive Design and Media Queries
    
    Everything you build should look good on all screens. 
\end{itemize}
\section{Sass - \emph{FE, FS}}
Sass is a CSS preprocessor at gives you more functionality in your styling. Sass offers things like variables, mixins, functions, nesting, etc. While Sass is optional, it is a useful tool. After learning CSS, it is easy to pick-up.

\section{CSS Frameworks - FE, \emph{FS} \small{(Choose one)}}

These UI frameworks offer pre-made classes to create elements on the fly. Menus, lists, alerts, etc. 
It's important to understand CSS before diving into a framework. 
Popular frameworks include: 
\begin{itemize}
    \item Tailwind CSS - low-level utility classes. Which means you can create your own cards. On the downside, your HTML has a ton of CSS classes. On the up, it's highly customizable and you don't really need to write any CSS.
    \item Bootstrap - CSS classes that give you alerts, cards, etc. A popular framework.
    \item Materialize - Based on Material Design.
    \item Bulma - Modular \& lightweight.
\end{itemize}

\section{UI Design - FE}

At a larger company, someone else is probably going to be doing the design work. But in general, it's important to know the basics of design. How large margins should be, what types of fonts to use, what sizes, etc.
\begin{itemize}
    \item Color and Contrast - Make sure text is readable.
    \item White Space - Spacing between elements.
    \item Scale - Sizing relative to other elements.
    \item Visual Hierarchy - Arrange in order of importance.
    \item Typography - Text typefaces, sizing, etc.
\end{itemize}

\section{JavaScript - FE, \emph{BE, FS}}
JS is extremely important for front-end/full stack web developers. It's the language of the client side. If you plan on doing back-end, just learning the basics of JS should be plenty. 

\begin{itemize}
    \item Basics - Variables, arrays, functions, loops, etc.
    \item DOM and Styling - Selecting and manipulating elements.
    \item Array Methods - foreach, map, filter, reduce, etc.
    \item JSON - JavaScript Object Notation
    \item HTTP Requests - Fetch API - GET, POST, PUT, DELETE
\end{itemize}
