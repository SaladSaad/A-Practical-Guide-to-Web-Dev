\section{Mobile Development \small{(Choose one)}}
More and more web developers are getting into mobile development with web-related technologies.

\begin{itemize}
    \item Flutter/Dart
    
    Open-source, user-interface, software developer kit (SDK). Uses the Dart Language. If you know JS, Dart is pretty easy to pick up.
    \item React Native
    
    If you already know React, might want to get into React Native. React Native is a framework for building native mobile apps using React.
    \item Ionic
    
    Javascript Based.
    \item Xamarin
    
    Allows you to build mobile apps with C\#.
    \item Kotlin
    
    Allows you to build native Android apps.
    \item Swift
    
    Allows you to build native iOS apps.
\end{itemize}

\section{Progressive Web Apps (PWA)}

Regular web apps built with HTML, CSS, and JS but run and feel like a native mobile app. Not just a responsive website but with a completely native feel as far as experience, layout, and functionality regardless of the device.

\begin{itemize}
    \item Built for all screen sizes.
    \item Offline content/Service Workers
    
    Offline content managed by service workers in the browser. Look into service workers if you want to get into PWAs.
    \item HTTPS
    \item Native experience (fast, engaging, splash screens, installable, etc).
\end{itemize}

\section{Desktop Apps with Web Technologies}

There are different web-technologies that can be used to create desktop apps. 
\begin{itemize}
    \item Electron
    
    The most popular. You can use JS to create desktop apps. Desktop apps built with Electron include Slack, Atom, VSCode.
    \item NW.js
    \item Python and Tkinter
\end{itemize}

\section{AI/Machine Learning}
A bit beyond the scope of web dev but are special situations where they can be beneficial to learn. Machine learning can be especially useful for Python developers.

\begin{itemize}
    \item Automation and Tools.
    \item Machine Learning APIs
    \item Understand user behavior/engagement/analytics
    \item Create code - GPT-3
\end{itemize}

\section{Web Assembly}
Efficient, low-level bytecode for the web. It's an "improvement" to JS, not a replacement. Meaning it can do much more powerful things in specific areas. 

\begin{itemize}
    \item Create extremely powerful web apps (games, video/image editing, etc)
    \item Can use languages such as C++ and Rust to compile to WASM. 
    \item AssemblyScript is a variant of TypeScript and it makes it easy to compile to WASM without learning a new language.
\end{itemize}

\section{Algorithms}
Algorithms may not seem productive however, they help your logic and critical thinking skills in ways you can't imagine.

\begin{itemize}
    \item Beginner algorithm questions: FizzBuzz, string reversals, array chunking, palindromes, anagrams, max character, etc.
    \item Popular challenge websites: Codewars, Project Euler, Coderbyte.
\end{itemize}

\section{Data structures (Primitive and non-primitive)}
Organizing and managing data efficiently so that we can perform specific operations efficiently. 

\begin{itemize}
    \item Popular data structures: Array, linked list, queue, stack, tree, graph, hash table.
\end{itemize}

\section{Software Design Patterns}

As you become a more advanced developer and write more intricate code, you might want to look into more specific software design patterns which are general reusable solutions to commonly occurring problems. Design patters are language agnostic, just like data structures and algorithms. 

\begin{itemize}
    \item Singleton
    \item Facade
    \item Bridge/adapter
    \item Strategy
    \item Observer
\end{itemize}